%______________________________________________________________________________________________________________________
% @brief	LaTeX2e Resume for Christian Ofenberg
\documentclass[margin,line]{Ofenberg_Resume}
\usepackage{hyperref}


%______________________________________________________________________________________________________________________
\begin{document}
\name{\Large Christian Ofenberg\ \ \ \ \ \ \ \ \ \ \ \ \ \ \ \ \ \ \ \ \ \ \ \ \ \ \ \ \ \ \ \ \ \ \ \ \ \ \ \ \ \ \ \ \ \ \ \ \ \ \ \ \ \ \ \ \ \ \ \ R\'{e}sum\'{e}}
\begin{resume}

	%__________________________________________________________________________________________________________________
	% Contact Information
	\section{\mysidestyle Contact\\Information}

	street											\hfill mobile: on request									\vspace{0mm}\\\vspace{0mm}%
	city										\hfill e-mail: on request	\vspace{0mm}\\\vspace{0mm}%
	Germany													\hfill Online: \url{http://www.ablazespace.de}	\vspace{0mm}\\\vspace{-4.5mm}%


	%__________________________________________________________________________________________________________________
	% Personal Information
	\section{\mysidestyle Personal\\Information}
	Born February 27th, 1982\\
	German citizen


	%__________________________________________________________________________________________________________________
	% Research Interests
	\section{\mysidestyle Research\\Interests}

	Realtime computer graphics, parallel computing, cross-platform development,\\
	build systems, flexible and extensible software architecture


	%__________________________________________________________________________________________________________________
	% Experience
	\section{\mysidestyle Experience}
	\textbf{Senior software developer at Promotion Software GmbH} \hfill \textbf{September 2012 -- June 2024}\vspace{-3mm}\\\vspace{-1mm}%
	\begin{list2}
		\item Lead in-house technology developer of the Quadriga Simulation Framework (QSF): Engine, graphics, tools and support for gameplay programing, artists and quality assurance
		\item Part of QSF was the conception and realization of a Qt based cooperative online editor which enabled the level designers to work together on EMERGENCY 5 maps at one and the same time, shipped as part of the EMERGENCY 5 modding SDK
		\item Technical coordination (conception, reviews, acceptance) during the EMERGENCY 5 development phase of up to 14 internal and external software developers
		\item Concrete projects: EMERGENCY 5, EMERGENCY 2016, EMERGENCY 2017, EMERGENCY 20 ( \url{https://store.steampowered.com/app/735280/EMERGENCY_20/} ), TEAMWORK research project ( \url{https://www.teamworkprojekt.de/} ), non-public research and client projects using AR (HoloLens) and VR (Oculus Rift and HTC Vive), free-to-play EMERGENCY ( \url{https://store.steampowered.com/app/850170/EMERGENCY/} ) using Unreal Engine 5 and custom C++ game servers
	\end{list2}\vspace{-1.5mm}
	\textbf{Unrimp - Free open-source 3D rendering project} \hfill \textbf{June 2012 -- January 2022}\vspace{-3mm}\\\vspace{-1mm}%
	\begin{list2}
		\item Project for personal fun and to be able to keep my graphics programming skills up-to-date
		\item Used to prototype a material and shader blueprint system as well as other technologies which enabled me to bring EMERGENCY 5 in the EMERGENCY 2016 edition from Direct3D 9 to Direct3D 11 while having limited development resources available to get the migration done in time
	\end{list2}\vspace{-1.5mm}
	\textbf{PixelLight - Free open-source 3D application framework} \hfill \textbf{September 2002 -- August 2012}\vspace{-3mm}\\\vspace{-1mm}%
	\begin{list2}
		\item One of the two lead developers
		\item Worked on basic data structures up to the C++ based Autodesk 3ds Max exporter
		\item Wrote documentation and supported the users of the technology
	\end{list2}\vspace{-1.5mm}
	\textbf{benntec Systemtechnik GmbH} \hfill \textbf{August 2008 -- August 2009}\vspace{-3mm}\\\vspace{-1mm}%
	\begin{list2}
		\item Six months internship semester, continued afterwards as a working student
		\item Attended meetings at Rheinmetall Defence Electronics (RDE) regarding 3D technologies
		\item Collaborated with the team and clients, identified requirements and presented realistic solutions
		\item Worked on the PixelLight based Compudent 3D dental patient advisory software
		\item Created a Java 3D port of PixelLight and provided technical assistance. The developed system was used in a simulator for fire fighting on board ships.
		\item Participated in the development of a PixelLight based tram-simulator prototype
		\item Supported the PixelLight based interactive Oerlikon product presentation with configuration and real-time 3D scenarios
	\end{list2}\vspace{-1.5mm}
	\textbf{Happy-Grafix Gbr} \hfill \textbf{March 2002 -- August 2003}\vspace{-3mm}\\\vspace{-1mm}%
	\begin{list2}
		\item Worked as a programmer on the commercial game project \emph{The Second Evolution} (cancelled)
		\item The \emph{Vulpine Vision} engine was used
	\end{list2}\vspace{-1.5mm}
	\textbf{\url{http://www.ablazespace.de}} \hfill \textbf{1995 -- 2002}\vspace{-3mm}\\\vspace{-1mm}%
	\begin{list2}
		\item Created open-source freeware games as a hobby
	\end{list2}\vspace{-1.5mm}


	%__________________________________________________________________________________________________________________
	% Software Development
	\section{\mysidestyle Software\\Development}

	Primarly:
	\begin{list2}
		\item C++, Unreal Engine 4 \& 5, OpenGL, Direct3D 9-12, Vulkan, OpenGL ES 3.0, GLSL, HLSL, Windows, Linux, macOS, Android (C API), C++ plugin development for Autodesk 3ds Max, CMake, Qt, OpenCL, Visual Studio, GCC, Doxygen, Subversion, Git, OOP and design patterns, Sentry for crash management, Steamworks
	\end{list2}\vspace{-1.5mm}

	During my work on the PixelLight project, I wrote plugins for:
	\begin{list2}
		\item FMOD, FMODEx, OpenAL, OpenGL and OpenGL ES 2.0, DirectX,
		Qt, Lua, Python, V8 JavaScript, AngelScript, MySQL, PostgreSQL, SQLite, Newton Game Dynamics, ODE, PhysX, Assimp, libRocket and SPARK
	\end{list2}\vspace{-1.5mm}
	Some of them as proof of concept. During my master thesis, I added volume rendering as a plugin.

	Further worked with:
	\begin{list2}
		\item C, Java, C\#, Pascal, Amiga Basic, Assembler, Nintendo Switch (UE5), Sony PlayStation 5 (UE5), PlayFab, Microsoft Azure, Jenkins together with Windows batch script and PowerShell script for e.g. automatic continuous delivery and the project management tool Hansoft to create tasks etc.
	\end{list2}\vspace{-1.5mm}


	%__________________________________________________________________________________________________________________
	% Education
	\section{\mysidestyle Education}
	\textbf{University of Applied Sciences}, W\"urzburg, Germany
	\begin{list2}
		\item Master of Science (MSc) \hfill \textbf{October 2010 --June 2012}\vspace{-3mm}\\\vspace{-1mm}%
		\item Bachelor of Engineering (B. Eng.) \hfill \textbf{October 2006 -- September 2010}\vspace{-3mm}\\\vspace{-1mm}%
	\end{list2}\vspace{-1.5mm}


	%__________________________________________________________________________________________________________________
	% Languages
	\section{\mysidestyle Languages}
	\begin{list2}
		\item German: Native
		\item English: Fluent
	\end{list2}\vspace{-1.5mm}


	%__________________________________________________________________________________________________________________
	% Personal Interests
	\section{\mysidestyle Personal\\Interests}
	\begin{list2}
		\item Software development related research in general and graphics in particular
		\item Sci-Fi/fantasy literature and video-games
		\item Hiking and jogging as well as other sport to relax
	\end{list2}\vspace{-1.5mm}


%______________________________________________________________________________________________________________________
\end{resume}
\end{document}


%______________________________________________________________________________________________________________________
% EOF

